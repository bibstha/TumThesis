\section{Research Questions and Methodology}
\label{sec:research_questions_and_methodology}

The following statements from \cite{pahl2014distributed} are of interest in the thesis:

\begin{enumerate}
  \item ``Having a statistic of the popularity of a model, it is likely that service developers in a crowdsourced development scenario orientate at the popularity as most popular context models are likely to be supported in most Smart Spaces in the world.''
  \item ``Collecting the error statistics and usage statistics can help providing users with feedback about the reliability and the use of a service which can be used to conclude about the quality and usefulness of a service.''
\end{enumerate}

We can now ask the following questions:

From \textbf{1}, 
\begin{itemize}
  \item[] \textbf{Q1} Is the assumption valid?
  \item[] \textbf{Q2} What happens if developers orientate or do not orientate towards the popularity of existing?
\end{itemize}

From \textbf{2},
\begin{itemize}
  \item[] \textbf{Q3} Is the assumption valid?
\end{itemize}

In both cases, we want to see how these statements can be validated. We are aware of two methodologies discussed below:

The modern scientific study of a phenomenon generally consists of three major approaches: theoretical, experimental and computational \cite{castiglione2006agent}.

\emph{Theoretical Study} for Smart Spaces have not been found. Most likely it is an dynamic and evolving at the moment.
% is not not so useful with AppStore study because AppStore is a moving and constantly changing structure with no formally defined precise behavior of humans: developers and users. Specially in an evolving field like smart spaces, theoretical study is unsuitable.

\emph{Experimental Studies (ES)} were done by \cite{pagano2013user} and \cite{chen2011predicting}. They make predictions based on the existing data by feeding them into different prediction models. 

\emph{Agent-based Modeling (ABM)} was used by \cite{lim2012successful} to simulate the behavior of users in App Ecosystem.

Both approaches have their advantages and drawbacks. Experimental studies are suited when existing facts based data are available.

In this thesis, we instead pick Agent-based Modeling because of the following reasons:

\begin{enumerate}
  \item The dynamics of entities association differs in S2Store compared to existing mobile app stores. So not having existing data suits ABM compared to ES \cite{castiglione2006agent}.
  \item ABM allows us to make variations in our assumptions and investigate the outcome \cite{castiglione2006agent}.
  \item ABM has been tested in economics, social sciences, ecology and many other fields which are relevant to ecosystem we are working with \cite{lim2012successful}.
\end{enumerate}