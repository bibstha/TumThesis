\section{Portability and Support for Collaborative Convergence}
\label{sec:portability}

DS2OS architecture by design allows portability of software services and smart device.

Services are written in Java and packaged in standard Java Archive (jar) format. Services are portable because they run in all the platforms where Java Runtime is available given that the required smart devices are available in the platform. 

Services are also portable in the sense that they are not tied to a specific type and model of a smart device. Instead a service is compatible with a range of smart devices that share a common abstract interface (Context Models) in the VSL. Communication between the VSL context model and the device is done with adaptation services.

The result of portability of software services and smart device abstraction means they can be shared easily among other developers and users. Sharing is done by publishing them in the S2Store.

Portability causes sharability and sharability causes popularity. Popularity of services and context models reduces duplication. This process is called convergence.

Due to convergence, it is expected that there exists one popular context model among many that is compatible with as many similar diverse devices as possible. The side effect of convergence causes future smart device manufacturers to make sure that their devices are also compatible. Eventually convergence leads to automatic standardization.

The whole process is driven through the open platform provided by S2Store. And every individual developer has a equal hand in the process.

\subsection*{Conclusion}

DS2OS architecture provides portability of software services and context models. Portability combined with uploading in S2Store makes collaborative convergence and automatic standardization.

