\chapter{Related Work}

Pahl presents \emph{Smart Space Store (S2Store)} which is an App Store for smart spaces. He argues that S2Store can be as revolutionary for smart space ecosystem as App Store was for Apple's ecosystem \cite{pahl2014distributed}.

An explanation of the dynamics inside an app store is presented in \cite{Jansen}. It lists the roles of individuals involved: users, developers, platform maintainers and how they interact with each other. It also compiles a list of required features and policies present in six existing Mobile app stores. It is interesting to see that Reputation Mechanisms like voting and app review are standard part of all of the mobile app stores. This is relevant to the thesis particularly because app stores interact with hundreds of thousands of developers \cite{lim2012successful} and reputation systems help to maintain an order in the system.

The benefits of Reputation Systems have been widely studied \cite{Akerlof1970} \cite{Axelrod1984} \cite{Resnick2000} \cite{farmer2010building}. The advantages and challenges in building Reputation Systems are presented in \cite{Resnick2000}. Different Reputation System models and their use case scenario are presented in \cite{farmer2010building}. The benefits of explicit rankings are well known but implicit rankings are not. Implicit rankings are a part of growing trend of measuring user behavior online to understand them better \cite{claypool2001inferring}. An example of effectiveness of implicit ranking algorithm is presented in \cite{girardello2010explicit}.

One of the major problems in building softwares for smart spaces in the diversity of smart devices. It is difficult to build a software for lamp that can work in majority of housing systems with different models of lamps. Cook et. al proposes a solution of all new smart devices to contain unique tags based on a consistent semantics \cite{Cook2012}. Pahl argues that building semantic ontology is a difficult proces given the existing variety devices at scale \cite{pahl2014distributed}. He proposes to use the dynamics of crowdsourcing this process. Existing studies prove empirically that a central software repository is very helpful for collaborative development irrespective of geographical position \cite{dabbish2012social}. Users are very likely to provide helpful feedback to developers through the app stores \cite{pagano2013user}. So, the crowdsourcing concept can be extended to collaboratively build an ontology of smart devices  based on popularity of their use \cite{pahl2014distributed}.

Simulation of an app store helps to observe the dynamic process inside the S2Store. Lim and Bentley \cite{lim2012successful} observed the correlation of using different strategies by developers to their success by simulating Apple's App Store. The simulation is called AppEco. It is an artificial agent-based model simulation that abstracts developers, users and apps in suitable models and adds them with behavioral assumptions. They have further used AppEco in different experiment to answer more questions. Thus AppEco has been used as a based model on top of which Smart Space Ecosystem (S2Eco) has been created as part of the thesis.

Entities involved in S2Eco: developers, users, software, context models, access groups have interconnections with each other. During simulation large number of entities are created. The structure of growth of graph has been studied in mathematics. When connections are chosen randomly, it creates a random graph. Random graphs do not depict the real nature of interconnection in software systems. According to LaBelle and Wallingford \cite{labelle2004inter} who studied the inter-package dependency networks in open source software, software system graphs have scale-free degree distributions, exhibit small-world effect and have assortative mixing \cite{newman2002assortative}. These properties have also been confirmed by simulations done in \cite{myers2003software}. It is desirable to have these properties in the simulation of S2Eco.
