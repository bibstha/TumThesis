\chapter{\abstractname}

App stores provide a central place to upload apps for mobile platforms. The introduction of Apple's App Store in 2008 enabled developers and users to exchange services by uploading apps through the App Store. Smart space platforms can receive similar benefits by the introduction of app stores for smart spaces. One of such app stores has been envisioned called ``Smart Space Store (S2Store)''. 

S2Store supports applications written on top of a framework for smart spaces known as ``Distributed Smart Space Orchestration System (DS2OS)''. Developers upload the applications written for DS2OS and users can download those applications. The problem in smart space environments is the diversity of devices and lack of standardization. S2Store can solve this problem by engaging users and developers with ranking mechanisms for uploaded applications. When significant amount of ranking activities occur, applications can be categorized according to their quality.

The thesis presents different mechanisms suitable for S2Store to support the process of convergence. It also presents a study of fitness of these mechanisms in the S2Store using a simulation called S2Eco. A detailed analysis of the elements of a smart space ecosystem is presented with the thesis.

With the results of the simulation, we show that the smart space ecosystem inside S2Store evolve with the interaction of developers and users. We see that due to the ranking mechanisms, quality of applications improve over time. We also see that despite the diversity of hardware devices and lack of standardization thereof, reputation mechanisms engage the crowd of developers and users to identify higher quality and better compatible interfaces. We therefore see the mechanisms for convergence in the smart space app store.
