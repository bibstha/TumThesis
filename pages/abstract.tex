\chapter{\abstractname}

Smart Spaces are an exciting area with plenty of existing research. Different advanced have been made in smart devices, network standards and protocols and software systems. Still, there are challenges to build software systems that can run in multitude of different enviroments with variety of different smart devices. There are also challenges to write interoperable software systems.

A similar problem existed for softwares written for mobile devices. But in the last ten years, Apple has democratized the software distribution landscape by introducing the App Store. App store provides a central place of software distribution with equal reach to developers and users from anywhere in the world. Democratization of software development for Spart Spaces is waiting to happen and has been envisioned by S2Store, an app store for Smart Spaces. 

S2Store supports applications written on top of a framework for smart spaces known as ``Distributed Smart Space Orchestration System (DS2OS''. S2Store and DS2OS do not define any default standards for hardwares devices or default communication protocols etc. However the major challenge for S2Store is to engage developers and users such that their activities lead towards convergence and thus emergence of standards selected by the community.

The thesis presents different mechanisms suitable for S2Store to support the process of convergence. A simulation of S2Store with selected reputation systems called S2Eco is created. It is then used to study the evolution of the smart space ecosystem and to confirm that the reputation mechanisms lead to desired converngence.
